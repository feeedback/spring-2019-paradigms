\section{Запросы посложнее}
\begin{frame}
	\tableofcontents[currentsection]
\end{frame}

\subsection{Группировка строк}
\begin{frame}
	\tableofcontents[currentsection,currentsubsection]
\end{frame}

\begin{frame}[fragile]{GROUP BY "--- картинка}
	\begin{center}
		\begin{minipage}{0.4\textwidth}
\begin{minted}{sql}
SELECT SUM(Pop), GovForm
FROM Country
WHERE ...
GROUP BY GovForm
HAVING ...
\end{minted}
		\end{minipage}
\vspace{10pt}

		\begin{tabular}{rllclll}
			\hline
			& Pop & GovForm & & SUM(Pop) & GovForm & \\\hline
			\multirow{6}{*}{$\xrightarrow{\t{WHERE}}$} & 618 & Republic & \rdelim\}{3}{20pt}[$\longrightarrow$] & \multirow{3}{*}{1134} & \multirow{3}{*}{Republic} & \multirow{6}{*}{$\xrightarrow{\t{HAVING}}$}\\
			& 455 & Republic \\
			& 61 & Republic \\
			& 650 & Monarchy & \rdelim\}{2}{20pt}[$\longrightarrow$] & \multirow{2}{*}{17425} & \multirow{2}{*}{Monarchy} & \\
			& 17364 & Monarch \\
			& \vdots & \vdots & & \vdots & \vdots & \\
		\end{tabular}
	\end{center}
\end{frame}

\begin{frame}{GROUP BY}
	\begin{itemize}
		\item Полезно, когда мы хотим посчитать какую-то статистику по подмножествам строк.
		\item Каждая агрегирующая функция работает только внутри группы.
		\item Например, суммарное население стран с разным политическим строем.
		\item Можно группировать по нескольким полям.
		\item Условие \t{WHERE} применяется до группировки.
		\item \t{ORDER BY} применяется после (так как группировка от порядка не зависит).
		\item В \t{SELECT} можно использовать только агрегирующие функции и колонки, по которым сделана группировка
			(иначе неясно, из какой строки выбирать).
			Некоторые СУБД ругаются, некоторые делают что-то.
		\item Можно дополнительно отфильтровать результаты после группировки при помощи \t{HAVING}.
	\end{itemize}
\end{frame}

\begin{frame}{Упражнения}
	Посчитайте:
	\begin{enumerate}
		\item Среднюю площадь страны в зависимости от формы правления.
		\item Средний квадрат площади страны в зависимости от формы правления.
		\item Количество стран с площадью до $500\,000$, от $500\,000$ до $1\,500\,000$ и так далее
			(надо сгруппировать по <<количеству миллионов>>).
	\end{enumerate}
\end{frame}
