\subsection{Вызовы функций}

\begin{frame}
	\tableofcontents[currentsection,currentsubsection]
\end{frame}

\begin{frame}[fragile]{Каррирование}
	\begin{itemize}
		\item Функция от двух аргументов $f(x, y)$ "--- то же самое, что функция от одного аргумента $f'(x)$,
			которая возвращает функцию от одного аргумента $g_x(y)=f(x, y)$.
		\item Такой переход называется \textit{каррированием}
		\item С точки зрения Haskell, функции типов \t{a -> b -> c} и \t{a -> (b -> c)} неотличимы.
		\item Все функции в Haskell "--- от одного аргумента, остальное "--- сахар.
		\item Из-за этого частичное применение и работает: \t{(3+)} и \t{(+) 3} равны.
		\item Можно явно написать функцию от двух аргументов (от кортежа):
\begin{minted}{haskell}
s (a, b) = a + b
s' = uncurry (+)
\end{minted}
	\end{itemize}
\end{frame}

\begin{frame}[fragile]{Способы определения функций}
	\begin{itemize}
		\item
\begin{minted}{haskell}
addTen x = x + 10
addTenToAll x = map addTen x
\end{minted}
		\item
\begin{minted}{haskell}
addTen = (+10)
addTenToAll = map f
\end{minted}
	\end{itemize}
\end{frame}

\begin{frame}[fragile]{Каррирование на практике}
	\begin{minted}{haskell}
addTen l = map (\x -> x + 10) l
addTen l = map (+10) l
addTen = map (+10)
addTen [1, 2, 3, 4]  -- [11, 12, 13, 14]

addTen' l' = map addTen l'
addTen' = map addTen
addTen' [[1, 2], [3, 4]] -- [[11, 12], [13, 14]]

map (map (+10)) [[1, 2], [3, 4]]
	\end{minted}
	Вывод: становится очень удобно комбинировать функции
\end{frame}

\begin{frame}[fragile]{Вызов функций}
	Доллар "--- оператор вызова функций без скобок:
\begin{minted}{haskell}
($) :: (a -> b) -> a -> b
(*2) (2 + 2)
(*2) $ 2 + 2
\end{minted}
\end{frame}

\begin{frame}[fragile]{Композиция функций-1}
	Точка "--- оператор композиции функций:
\begin{minted}{haskell}
(.) :: (b -> c) -> (a -> b) -> (a -> c)
-- Приоритет низкий, выполняется после применения функций:
((+5) . (*3)) 10
\end{minted}
	Читать надо справа налево:
\begin{minted}{haskell}
(concat . map words . lines) "1 2\n10 11\n"
\end{minted}
	Композиции функций в каком-то смысле лучше, чем свои вспомогательные функции или явные вызовы.
\end{frame}

\begin{frame}[fragile]{Композиция функций-2}
	В пакете \t{Control.Arrow} есть оператор \t{>>>} для композиции функций (и не только):
\begin{minted}{haskell}
-- Упрощённый тип
(>>>) :: (a -> b) -> (b -> c) -> (a -> c)
-- Приоритет низкий
(+10) >>> (*2) >>> (+3) $ 100
\end{minted}
	Читать надо слева направо:
\begin{minted}{haskell}
(lines >>> map words >>> concat) "1 2\n10 11\n"
\end{minted}
	Композиции функций в каком-то смысле лучше, чем свои вспомогательные функции или явные вызовы.
\end{frame}
